%% start of file `template.tex'.
%% Copyright 2006-2015 Xavier Danaux (xdanaux@gmail.com).
%
% This work may be distributed and/or modified under the
% conditions of the LaTeX Project Public License version 1.3c,
% available at http://www.latex-project.org/lppl/.

% IMPORTANT: with multibib, remember to bibtex books.aux etc so that .bbl and .blg are generated.

\documentclass[11pt,a4paper,sans]{moderncv}        % possible options include font size ('10pt', '11pt' and '12pt'), paper size ('a4paper', 'letterpaper', 'a5paper', 'legalpaper', 'executivepaper' and 'landscape') and font family ('sans' and 'roman')

% moderncv themes
\moderncvstyle{banking}                             % style options are 'casual' (default), 'classic', 'banking', 'oldstyle' and 'fancy'
\moderncvcolor{red}                               % color options 'black', 'blue' (default), 'burgundy', 'green', 'grey', 'orange', 'purple' and 'red'
%\renewcommand{\familydefault}{\sfdefault}         % to set the default font; use '\sfdefault' for the default sans serif font, '\rmdefault' for the default roman one, or any tex font name
%\nopagenumbers{}                                  % uncomment to suppress automatic page numbering for CVs longer than one page

% character encoding
\usepackage[utf8]{inputenc}    
\usepackage[english,french]{babel} 
                   % if you are not using xelatex ou lualatex, replace by the encoding you are using

% adjust the page margins
\usepackage[scale=0.75]{geometry}
%\setlength{\hintscolumnwidth}{3cm}                % if you want to change the width of the column with the dates
%\setlength{\makecvtitlenamewidth}{10cm}           % for the 'classic' style, if you want to force the width allocated to your name and avoid line breaks. be careful though, the length is normally calculated to avoid any overlap with your personal info; use this at your own typographical risks...
\usepackage{ragged2e}
% personal data

%\usepackage{fontspec}


\name{Alberto}{Lumbreras}
\title{CV}                               % optional, remove / comment the line if not wanted
%\address{12 rue de la Blanchisserie}{31500 Toulouse}{France} 
\phone[mobile]{+33~634~266~794}       
%\phone[fixed]{+33~531~986~660}
\email{alberto.lumbreras@irit.fr}                               \homepage{www.albertolumbreras.net}                          \social[linkedin]{albertolumbreras}                        
\social[twitter]{albertlumbreras}                            
\social[github]{alumbreras}
%\quote{\texttt{Status:} Visiting PhD student  at IRIT (Feb 2016 -- present)}  
% optional, remove / comment the line if not wanted

% bibliography adjustements (only useful if you make citations in your resume, or print a list of publications using BibTeX)
%   to show numerical labels in the bibliography (default is to show no labels)
%\makeatletter\renewcommand*{\bibliographyitemlabel}{\@biblabel{\arabic{enumiv}}}\makeatother

%   to redefine the bibliography heading string ("Publications")
%\renewcommand{\refname}{Articles}

% bibliography with mutiple entries
\usepackage{multibib}
\newcites{book,article,proceeding,subproceeding,proceedingnational, patent,talk}{{Books},{Journals},{Conferences and Workshops (international)},{Conferences and Workshops (international)},{Conferences and Workshops (national)},{Patents},{Seminars and Posters}}
%----------------------------------------------------------------------------------
%            content
%----------------------------------------------------------------------------------
\begin{document}

\selectlanguage{french}
%-----       letter    ------------------------
% recipient data
\recipient{Université Jean Jaurès}{Toulouse, CNU 27}
\date{22 mars de 2016}
\opening{Chère Madame, Cher Monsieur,}
\closing{Veuillez agréer, Madame, Monsieur, l’expression de mes respectueuses salutations,}
%\enclosure[Attached]{curriculum vit\ae{}}          % use an optional argument to use a string other than "Enclosure", or redefine \enclname

\clearpage


\makecvtitle
\section{Personal information}
\cvitem{Date and place of birth}{April 4th 1981, Barcelona}
\cvitem{Nationality}{Spanish}
\section{Education}
\cventry{Lyon}{Université de Lyon 2}{PhD, Computer Science}{Feb 2013--2016}{}{Industrial PhD (CIFRE) in Technicolor R\&D (Rennes).\\
\underline{Supervisors}:\\
Bertrand Jouve (CNRS, U. Toulouse 2. Previously: U. Lyon 2), \\
Julien Velcin (Université Lyon 2, ERIC Lab), \\Marie Guégan (Technicolor)}  % arguments 3 to 6 can be left empty
\cventry{Barcelona}{Universitat Politècnica de Catalunya}{Master, Artificial Intelligence}{2009--2012}{}{(2-year programme. Score: 7,69/10)}
\cventry{Barcelona}{Universitat Politècnica de Catalunya}{BSc, Telecommunication Engineering}{1999--2005}{}{(5-year programme)}

\subsection{Summer Schools}
\cventry{Luchon}{\url{http://www.quantware.ups-tlse.fr/ecoleluchon2014/}}{Network analysis and applications}{August 2014}{}{}
\cventry{Reykjavik}{\url{http://mlss2014.hiit.fi/}}{Machine Learning Summer School}{May 2014}{}{}
\cventry{Les Houches}{\url{http://leshouches2014.weebly.com/}}{Structure and dynamics of complex networks}{April 2014}{}{}
\cventry{Lisbon}{\url{http://lxmls.it.pt/2011/Home.html}}{Lisbon Machine Learning School}{July 2011}{}{}

\section{PhD Thesis}
\cvitem{title}{\emph{
Automatic Role Detection in Online Forums.
}}
\cvitem{supervisors}{Bertrand Jouve (CNRS, U. Toulouse 2. Previously: U. Lyon 2), \\
Julien Velcin (Université Lyon 2, ERIC Lab), Marie Guégan (Technicolor)}

\cvitem{description}{The aim of the thesis is to develop new Machine Learning and Social Network Analysis techniques to analyse the global and individual dynamics of online forums, to automatically find the different latent roles of participants and to study whether online social roles can be used as predictors for some aspects of users behaviour. Because behaviors in forums are primarily made of conversations, we focus our attention on they how users conversate.
\vspace{0.1cm}  
\\
We propose three methods. Our first method for the detection of roles based on conversational structures. We apply different notions of neighborhood for posts in tree graph and we use these neighborhoods to detect groups of users that tend to participate in the same type of conversation.
\vspace{0.1cm}  
\\
Our second method is based on stochastic models of growth for conversation threads. Building upon these generative models we propose a method to find groups of users that tend to reply to the same type of posts. We show that, while we are able to find clusters of users based on their past behaviors, there is no evidence that these behaviors are predictive of future behaviors. 
\vspace{0.1cm}  
\\
In out last method we integrate the type of data used in the two previous methods (feature-based and behavioral or functional-based) and show that we can find clusters using fewer examples. The model exploits the idea that users with similar features have similar behaviors. Not only the method integrates inputs of different nature, adding more consistency to the clusters, but it tackles the problem of clustering users when most users have only participated a few times.}

\section{Master's Thesis}
\cvitem{title}{\emph{Towards Trust-Aware Recommender Systems}}
\cvitem{supervisor}{Ricard Gavaldà}
\cvitem{description}{Recommender   systems   have   been strongly   researched  within  the  last  decade.  With  the  arising  and  popularization   of   digital   social   networks   a   new   field   has   been opened  for  social recommendations.  We  propose  a  way  to  infer trust from  Twitter  interactions  and  to  compute  trust between  distant users. Our method  is  based  on  Markov chains,  which  makes it  simple  to  be  implemented  and computationally efficient.  We study the properties of this trust metric and study its application in a recommender system of tweets.}

\section{Experience}
\subsection{Academic supervision}
\cventry{Lyon}{Université de Lyon 2 -- Technicolor}{Master's Thesis co-Supervisor}{2015}{}{}{\textit{Master in Data Mining and Knowledge Management.}\newline{}Co-supervised the Master's Thesis of Zara Alaverdyan at Technicolor, entitled "Language-based analysis in online forums" (supervisor: Julien Velcin).}

\subsection{Teaching}%
\small{(Due to my CIFRE contract I could not teach at the university during my thesis. I have started teaching as soon as my CIFRE contract expired.)}\\

\cventry{Toulouse}{Université Paul Sabatier}{Vacataire}{March-April-May 2016}{}{Systèmes d'information et applications Web (TP, 20h) \\Web applications with PHP and MySQL.}

\cventry{Toulouse}{IUT Rangueil}{Vacataire}{March-April-May 2016}{}{
UML design with Modelio (TP, 18h)}

\cventry{Barcelona}{Epsilon Formación}{Teacher}{2000}{\url{http://www.epsilon-formacion.com/}}{Elaboration and teaching of courses for first years students of Telecommunication Engineering (TD, 40h)}


\subsection{R\&D}

\cventry{Barcelona}{Universitat Politècnica de Catalunya}{Research Assistant}{2010--2012}{}{
\begin{itemize}
\item \textbf{Sentiment Analysis and Stream Mining}: Adapted the popular MOA software (stream data mining) to be used with particular sentiment analysis methods in Twitter streams and to show the results in a friendlier interface. I worked in close collaboration with Albert Bifet, main developer of MOA.
\item \textbf{Recommender Systems} (Master's Thesis).
\end{itemize}
}

\cventry{Barcelona}{Telefónica R\&D}{R\&D Engineer}{2004--2010}{}{%
Main projects:%
\begin{itemize}%
\item \textbf{Recommender Systems \& User Profiling (2009--2010)}: worked in the back-end of a platform to integrate all recommendations and user profiling services in Telefónica.
\item \textbf{Movisphere (corporate spin-off) (2008-2009)}: as a part of a two-people team,  moved to Vancouver for three months to build a start-up founded by Telefónica in collaboration with the Center for Digital Media. Movisphere was a multi-platform (mobile, TV, PC) service of \textit{casual games}. The project received the following awards: 
  \begin{itemize}%
  \item Best New Business Opportunity at Telefónica's Second Annual Research Fair.\\ \url{http://thecdm.ca/projects/archives/movisphere}
  \item Honoree in the Integrated Mobile Project category at the 13th Webby Awards.\\ \url{http://tinyurl.com/webbyawards-movisphere}
  \end{itemize}
\item \textbf{Peer-to-Peer Networks (2006--2007)}: worked on P2P (bluetooth and IPTV) content distribution, prototypes development and analysis of feasibility for integration in Telefónica platform.
\item \textbf{Expert Systems (2004--2006)}: implementation of an expert system that monitors and automatically corrects failures in a tier center of a grid network (EGEE - Enabling Grids for E-Science in Europe).
\end{itemize}}


\section{Languages}
\cvdoubleitem{French}{fluent}{Spanish}{Native}
\cvdoubleitem{English}{fluent (TOEFL iBT: 104)}{Catalan}{Native}

% Publications from a BibTeX file without multibib
%  for numerical labels: \renewcommand{\bibliographyitemlabel}{\@biblabel{\arabic{enumiv}}}% CONSIDER MERGING WITH PREAMBLE PART
%  to redefine the heading string ("Publications"): \renewcommand{\refname}{Articles}

%\clearpage
% Publications from a BibTeX file using the multibib package
\section{Publications}
\begin{itemize}
\item Books: 1
\item International workshops: 2
\item National conferences: 1
\item Patents: 1
\item Journals: 1
\end{itemize}

\nocitearticle{Lumbreras2016}
\bibliographystylearticle{plain}
\bibliographyarticle{mypublications}

\nocitebook{toret2015}
\bibliographystylebook{plain}
\bibliographybook{mypublications}


\nociteproceeding{Lumbreras2012a, Lumbreras2013a}
\bibliographystyleproceeding{plain}
\bibliographyproceeding{mypublications}   

\nociteproceedingnational{Lumbreras2013b}
\bibliographystyleproceedingnational{plain}
\bibliographyproceedingnational{mypublications}   

\nocitetalk{Lumbreras2016MCMC, Lumbreras2016MIT, Lumbreras2015LARCA, Lumbreras2015d, Lumbreras2015MOREHIST, Lumbreras2015IRIT, Toret2012}
\bibliographystyletalk{plain}
\bibliographytalk{mypublications}  

\section{Publications submitted}

\nocitesubproceeding{Lumbreras2016ASONAM}
\bibliographystylesubproceeding{plain}
\bibliographysubproceeding{mypublications}

\section{Patents}
%\nocitepatent{Lumbreras2015}
%\bibliographystylepatent{plain}
%\bibliographypatent{mypublications}  
Guégan M., Jouve B. Lumbreras A., and Velcin J. Methods and systems for clustering-based
recommendations, July 2015. [EU Filled Patent, number 15306168.4].

\section{Conferences and workshops organization}
\cventry{Toulouse}{(Institut de Systèmes Complexes de Toulouse)}{Dynamic Networks workshop}{2016}{}{\url{http://isc-t.fr/language/fr/evenements/?event_id1=5}\\Hotel and trip reservation for invited speakers; communications (mailing, website announcements), agenda.}
\cventry{Barcelona}{\url{http://www.ecmlpkdd2010.org/index1f1e.html}}{ECML-PKDD}{2010}{}{Local logistics, reception and assistance for speakers during the conference.}



\section{References}
\subsection{References in academy}
  \begin{itemize}
  \item Bertrand Jouve (CNRS, U. Toulouse 2) \hfill jouve@univ-tlse2.fr                            
  \item Julien Velcin (U. Lyon 2)\hfill velcin@univ-lyon2.fr
  \item Marie Guégan (Technicolor)\hfill marie.guegan@technicolor.com
  \item Ricard Gavaldà (U. Politècnica de Catalunya)\hfill gavalda@cs.upc.edu
  \item Albert Bifet (Télécom ParisTech) \hfill albert@albertbifet.com
  \end{itemize}
\subsection{References in industry (Telefónica R\&D)}
\begin{itemize}
\item Enrique Garcia Illera (Direct Supervisor) \hfill egi@tid.es 
\item Carlos Domingo (Lab Director) \hfill \url{https://about.me/carlosdomingo}
\item Xavier Amatriain (co-worker, Senior Researcher) \hfill \url{http://xavier.amatriain.net/}
\end{itemize}




\end{document}


%% end of file `template.tex'.
