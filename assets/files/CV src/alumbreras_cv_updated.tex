%% start of file `template.tex'.
%% Copyright 2006-2015 Xavier Danaux (xdanaux@gmail.com).
%
% This work may be distributed and/or modified under the
% conditions of the LaTeX Project Public License version 1.3c,
% available at http://www.latex-project.org/lppl/.

% IMPORTANT: with multibib, remember to bibtex books.aux etc so that .bbl and .blg are generated.

\documentclass[11pt,a4paper,sans]{moderncv}        % possible options include font size ('10pt', '11pt' and '12pt'), paper size ('a4paper', 'letterpaper', 'a5paper', 'legalpaper', 'executivepaper' and 'landscape') and font family ('sans' and 'roman')

% moderncv themes
\moderncvstyle{banking}                             % style options are 'casual' (default), 'classic', 'banking', 'oldstyle' and 'fancy'
\moderncvcolor{red}                               % color options 'black', 'blue' (default), 'burgundy', 'green', 'grey', 'orange', 'purple' and 'red'
%\renewcommand{\familydefault}{\sfdefault}         % to set the default font; use '\sfdefault' for the default sans serif font, '\rmdefault' for the default roman one, or any tex font name
%\nopagenumbers{}                                  % uncomment to suppress automatic page numbering for CVs longer than one page

% character encoding
\usepackage[utf8]{inputenc}    
\usepackage[english,french]{babel} 
                   % if you are not using xelatex ou lualatex, replace by the encoding you are using

% adjust the page margins
\usepackage[scale=0.75]{geometry}
%\setlength{\hintscolumnwidth}{3cm}                % if you want to change the width of the column with the dates
%\setlength{\makecvtitlenamewidth}{10cm}           % for the 'classic' style, if you want to force the width allocated to your name and avoid line breaks. be careful though, the length is normally calculated to avoid any overlap with your personal info; use this at your own typographical risks...
\usepackage{ragged2e}
% personal data

%\usepackage{fontspec}


\name{Alberto}{Lumbreras}
\title{CV}                               % optional, remove / comment the line if not wanted
%\address{12 rue de la Blanchisserie}{31500 Toulouse}{France} 
\phone[mobile]{+33~634~266~794}       
%\phone[fixed]{+33~531~986~660}
\email{alberto.lumbreras@irit.fr}                               \homepage{www.albertolumbreras.net}                          \social[linkedin]{albertolumbreras}                        
\social[twitter]{albertlumbreras}                            
\social[github]{alumbreras}                              
%\quote{\texttt{Status:} Visiting PhD student  at IRIT (Feb 2016 -- present)}  
% optional, remove / comment the line if not wanted

% bibliography adjustements (only useful if you make citations in your resume, or print a list of publications using BibTeX)
%   to show numerical labels in the bibliography (default is to show no labels)
%\makeatletter\renewcommand*{\bibliographyitemlabel}{\@biblabel{\arabic{enumiv}}}\makeatother

%   to redefine the bibliography heading string ("Publications")
%\renewcommand{\refname}{Articles}

% bibliography with mutiple entries
\usepackage{multibib}
\newcites{book,article,proceeding,subproceeding,proceedingnational, patent,talk}{{Books},{Journals},{Conferences and Workshops (international)},{Conferences and Workshops (international)},{Conferences and Workshops (national)},{Patents},{Seminars and Posters}}
%----------------------------------------------------------------------------------
%            content
%----------------------------------------------------------------------------------
\begin{document}

\selectlanguage{french}

%\input{letter_cnrs.tex}

\makecvtitle
\section{Personal information}
\cvitem{Date and place of birth}{April 4th 1981, Barcelona}
\cvitem{Nationality}{Spanish}
\section{Education}
\cventry{Lyon}{Université de Lyon 2}{PhD, Computer Science}{Feb 2013--2016}{}{Industrial PhD (CIFRE) in Technicolor R\&D (Rennes).\\
\underline{Supervisors}:\\
Bertrand Jouve (CNRS, U. Toulouse 2. Previously: U. Lyon 2), \\
Julien Velcin (Université Lyon 2, ERIC Lab), \\Marie Guégan (Technicolor)}  % arguments 3 to 6 can be left empty
\cventry{Barcelona}{Universitat Politècnica de Catalunya}{Master, Artificial Intelligence}{2009--2012}{}{(2-year programme. Score: 7,69/10)}
\cventry{Barcelona}{Universitat Politècnica de Catalunya}{BSc, Telecommunication Engineering}{1999--2005}{}{(5-year programme)}

\subsection{Summer Schools}
\cventry{Luchon}{\url{http://www.quantware.ups-tlse.fr/ecoleluchon2014/}}{Network analysis and applications}{August 2014}{}{}
\cventry{Reykjavik}{\url{http://mlss2014.hiit.fi/}}{Machine Learning Summer School}{May 2014}{}{}
\cventry{Les Houches}{\url{http://leshouches2014.weebly.com/}}{Structure and dynamics of complex networks}{April 2014}{}{}
\cventry{Lisbon}{\url{http://lxmls.it.pt/2011/Home.html}}{Lisbon Machine Learning School}{July 2011}{}{}
\newpage
\section{Research experience (2004-2012)}
%\subsection{R\&D}
\cventry{Barcelona}{Telefónica Investigación y Desarrollo (Telefónica Research)}{Research Engineer}{2004--2010}{}{%
I worked for 6 years in the research branch of Telefónica (multinational broadband and telecommunications provider) as a Research Engineer. As such, I have been involved in European research projects (founded by the FP7 program) as well as in internal research projects, mainly for the Internet and the Mobile divisions of the company. My tasks ranged from the design and implementation of algorithms to the writing of project proposals and coordination of small research teams in exploratory areas.\\
Some of these projects are:%
\begin{itemize}%
\item \textbf{Expert Systems (2004--2006)}: design and implementation of an expert system that monitors and automatically corrects failures in a tier center of a Europan grid computing network for scientific research, coordinated by the CERN (Conseil Européen pour la Recherche Nucléaire) %(EGEE - Enabling Grids for E-Science in Europe).
\item \textbf{Peer-to-Peer Networks (2006--2007)}: design and implementation of P2P (bluetooth and IPTV) content distribution, prototypes development and technical analysis of feasibility for its integration in the Telefónica platform.
\item \textbf{Movisphere (corporate spin-off) (2008-2009)}: as a part of a two-people team,  I moved to Vancouver for three months to create a spin-off founded by Telefónica in collaboration with the Center for Digital Media. Movisphere was a multi-platform (mobile, TV, PC) service of \textit{casual games}. The project received the following awards: 
  \begin{itemize}%
  \item Best New Business Opportunity at Telefónica's Second Annual Research Fair.\\ \url{http://thecdm.ca/projects/archives/movisphere}
  \item Honoree in the Integrated Mobile Project category at the 13th Webby Awards.\\ \url{http://tinyurl.com/webbyawards-movisphere}
  \end{itemize}
\item \textbf{Recommender Systems \& User Profiling (2009--2010)}: I worked in a platform to integrate all recommendations and user profiling services in Telefónica.
\end{itemize}}

\cventry{Barcelona}{Universitat Politècnica de Catalunya}{Research Engineer}{2010--2012}{}{
\textit{Laboratory for Relational Algorithmics, Complexity and Learning.}\\
I developed and implemented algorithms and system prototypes for data mining, sentiment analysis and recommender systems. I also helped in the organization of workshops and conferences hosted by our research lab. I was also involved in technology transfer projects to help industrial partners in adapting the algorithms developed in the lab. The following are two of the main projects:
\begin{itemize}
\item \textbf{Sentiment Analysis and Stream Mining}: I adapted the popular MOA software (stream data mining) to be used with particular sentiment analysis methods in social networks streams. I worked together with the researchers at the University of Waikato (New Zealand) who created MOA.  %I worked in close collaboration with Albert Bifet, main developer of MOA.
\item \textbf{Recommender Systems}: I designed and implemented a prototype of a content recommender system for social networks. %(Master's Thesis).
\end{itemize}
}
\cventry{Rennes}{Technicolor R\&I}{Research Engineer -- Thèse  CIFRE}{2013--2016}{}{
In the context of a CIFRE thesis (industrial thesis) I worked as a Research Engineer in the research division of Technicolor (multinational specialized in digital video and image technologies). My thesis was part of a project that aimed to analyze online movie-related forums. 
}
\section{Teaching experience}
\subsection{Academic supervision}
\cventry{Lyon}{Université de Lyon 2 -- Technicolor}{Master's Thesis co-Supervisor}{2015}{}{}{\textit{Master in Data Mining and Knowledge Management.}\newline{}Co-supervised the Master's Thesis of Zara Alaverdyan at Technicolor, entitled "Language-based analysis in online forums" (supervisor: Julien Velcin).}

\subsection{Teaching}%
\cventry{Toulouse}{Université Paul Sabatier}{Vacataire}{March-April-May 2016}{}{Systèmes d'information et applications Web (TP, 24h) \\Web applications with PHP and MySQL.}

\cventry{Toulouse}{IUT Rangueil}{Vacataire}{March-April-May 2016}{}{
UML design with Modelio (TP, 18h)}

\cventry{Barcelona}{Epsilon Formación}{Teacher}{2000}{\url{http://www.epsilon-formacion.com/}}{Elaboration and teaching of courses for first years students of Telecommunication Engineering (TD, 40h)}

\section{Master's Thesis (2012)}
\cvitem{title}{\emph{Towards Trust-Aware Recommender Systems}}
\cvitem{supervisor}{Ricard Gavaldà}


\section{PhD Thesis (2013-2016)}
\cvitem{title}{\emph{
Automatic Role Detection in Online Forums.}}
\cvitem{supervisors}{Bertrand Jouve (CNRS, U. Toulouse 2. Previously: U. Lyon 2), \\
Julien Velcin (Université Lyon 2, ERIC Lab), Marie Guégan (Technicolor)}


\section{Languages}
\cvdoubleitem{French}{fluent}{Spanish}{Native}
\cvdoubleitem{English}{fluent (TOEFL iBT: 104)}{Catalan}{Native}

% Publications from a BibTeX file without multibib
%  for numerical labels: \renewcommand{\bibliographyitemlabel}{\@biblabel{\arabic{enumiv}}}% CONSIDER MERGING WITH PREAMBLE PART
%  to redefine the heading string ("Publications"): \renewcommand{\refname}{Articles}

%\clearpage
% Publications from a BibTeX file using the multibib package
\section{Publications}
\begin{itemize}
\item Books: 1
\item International workshops: 2
\item National conferences: 1
\item Patents: 1
\item Journals: 1
\end{itemize}

\nocitearticle{Lumbreras2016}
\bibliographystylearticle{plain}
\bibliographyarticle{mypublications}

\nocitebook{toret2015}
\bibliographystylebook{plain}
\bibliographybook{mypublications}


\nociteproceeding{Lumbreras2012a, Lumbreras2013a}
\bibliographystyleproceeding{plain}
\bibliographyproceeding{mypublications}   

\nociteproceedingnational{Lumbreras2013b}
\bibliographystyleproceedingnational{plain}
\bibliographyproceedingnational{mypublications}   

\nocitetalk{Lumbreras2016MCMC, Lumbreras2016MIT, Lumbreras2015LARCA, Lumbreras2015d, Lumbreras2015MOREHIST, Lumbreras2015IRIT, Toret2012}
\bibliographystyletalk{plain}
\bibliographytalk{mypublications}  

%\section{Publications submitted}
%\nocitesubproceeding{Lumbreras2016ASONAM}
%\bibliographystylesubproceeding{plain}
%\bibliographysubproceeding{mypublications}

\section{Patents}
%\nocitepatent{Lumbreras2015}
%\bibliographystylepatent{plain}
%\bibliographypatent{mypublications}  
Guégan M., Jouve B. Lumbreras A., and Velcin J. Methods and systems for clustering-based
recommendations, July 2015. [EU Filled Patent, number 15306168.4].

\section{Conferences and workshops organization}
\cventry{Toulouse}{(Institut de Systèmes Complexes de Toulouse)}{Dynamic Networks workshop}{2016}{}{\url{http://isc-t.fr/language/fr/evenements/?event_id1=5}\\Hotel and trip reservation for invited speakers; communications (mailing, website announcements), agenda.}
\cventry{Barcelona}{\url{http://www.ecmlpkdd2010.org/index1f1e.html}}{ECML-PKDD}{2010}{}{Local logistics, reception and assistance for speakers during the conference.}



\section{References}
\subsection{References in academy}
  \begin{itemize}
  \item Bertrand Jouve (CNRS, U. Toulouse 2) \hfill jouve@univ-tlse2.fr                            
  \item Julien Velcin (U. Lyon 2)\hfill velcin@univ-lyon2.fr
  \item Marie Guégan (Technicolor)\hfill marie.guegan@technicolor.com
  \item Ricard Gavaldà (U. Politècnica de Catalunya)\hfill gavalda@cs.upc.edu
  \item Albert Bifet (Télécom ParisTech) \hfill albert@albertbifet.com
  \end{itemize}
\subsection{References in industry (Telefónica R\&D)}
\begin{itemize}
\item Enrique Garcia Illera (Direct Supervisor) \hfill egi@tid.es 
\item Carlos Domingo (Lab Director) \hfill \url{https://about.me/carlosdomingo}
\item Xavier Amatriain (co-worker, Senior Researcher) \hfill \url{http://xavier.amatriain.net/}
\end{itemize}


\end{document}


%% end of file `template.tex'.
