\documentclass[paper=a4,fontsize=11pt]{temp} % KOMA-article class					
% https://xamat.github.io/cv.pdf		
\usepackage{enumitem}
\usepackage[none]{hyphenat}
%\usepackage{hyperref}
\usepackage{fontawesome}

    
\begin{document}

\begin{minipage}[t]{.65\linewidth}
   \MyName{Alberto~Lumbreras}{Researcher, PhD | Machine Learning | Data Science}
\end{minipage}%  
\begin{minipage}[t]{0.3\linewidth}  
\raggedright
\small{\faEnvelopeO}~alberto.lumbreras@gmail.com   
\small{\faGlobe}~http://albertolumbreras.net    
\small{\faGithub}~http://github.com/alumbreras
\small{\faPhone}~(+33) 634 26 67 94
\normalsize
 %\par
 \vspace{0.3cm} 

\end{minipage}
\par
\begin{minipage}{0.9\linewidth}
\MySummary{
\textit{Researcher with more than 10 years of international experience, both in industry and academia. Oriented to applied research, from social network analysis to recommender systems, with a solid background in statistical machine learning.}}
\end{minipage}

%\par
%\textit{Research areas}: Machine Learning, Bayesian statistics, Social Network Analysis, Recommender Systems.

 \vspace{0.1cm} 
\NewPart{Experience}{}
\noindent

\workEntry{Centre National de la Recherche Scientifique}{2016 -}{Postdoctoral Researcher}{
%\hspace{0cm}
\begin{itemize}[leftmargin=*]
\item Novel models and statistical methods (Monte Carlo algorithms, Variational Inference) for Machine Learning, specially for matrix factorization with applications such as text semantic analysis, recommender systems or social network analysis. 
\item 2 R packages for matrix factorization. 
\item 1 PhD thesis co-supervised.
\end{itemize}
} {IMG/CNRS}{Toulouse (France)}


\sepspace
\workEntry{LinkProved}{2016 -}{Data Science Consultant}{
\begin{itemize}[leftmargin=*]
\item Consulting for a start-up focused on product recommendations. Technical orientation and prototyping in Python for topic detection, user modeling and recommender systems.
\end{itemize}}{Toulouse }{Toulouse (France)}


\sepspace

\workEntry{Technicolor R\&D}{2013 - 2016}{Research Engineer/ Industrial PhD}{
\begin{itemize}[leftmargin=*]
\item Analysis and predictive models for user behaviors in online forums. Design and implementation of novel machine learning algorithms.
\item 1 patent application as main author and several applications as secondary author.
\end{itemize}} {IMG/Technicolor}{Rennes (France)}

\vspace{0.25cm}
\workEntry{Université Paul Sabatier}{\color{white}*}{Teaching Assistant}{\begin{itemize}[leftmargin=*]
\item Teaching in the practical sessions of two Computer Science courses: one on databases; another on UML, Object Oriented Programming, Test-Driven Development and Java. 
\item 1 master thesis supervised (Master in Data Mining and Knowledge Extraction).
\end{itemize}
} {IMG/CNRS}{Toulouse (France)}

\sepspace

\workEntry{Universitat Politècnica de Catalunya}{2010-2012}{
Research Engineer}{
Laboratory of Relational Algorithmics, Complexity and Learning (LARCA)
\begin{itemize}[leftmargin=*]
\item Novel algorithms for social network analysis, web mining, recommender systems, sentiment analysis, and stream data mining.
\item Industrial partnership (Datknosys): Implemented a stream mining solution for sentiment analysis in Twitter. The solution was based in the popular MOA software. Worked in collaboration with the MOA authors and with the Datknosys team to integrate this solution into their platform. 
\end{itemize}}{IMG/UPC}{Barcelona (Spain)}


\sepspace

\workEntry{Telefónica Research}{2004 - 2010}{Research Engineer}{
\begin{itemize}[leftmargin=*]
\item Participation in all phases of research and innovation projects, both internal and in collaboration with several Telefónica Business Units. Projects ranged from mobile peer-to-peer to recommender systems. 
\item Lead some small teams (up to 5 people).
\item Participation in multiple European projects with international partners.
\item 13th Webby Awards (2009): Best New Business Opportunity in the Integrated Mobile area. Lead a small team of developers in Vancouver, in collaboration with the Center of Digital Media, and elaborated a business plan for a spin-off. We were honored with a Webby Award. (\url{https://www.webbyawards.com/winners/movisphere/})
\item  Member of Telefónica's Ambassadors for Innovation, an internal network of employees which aimed to boost innovation in the company.
\end{itemize}
}{IMG/telefonica_cuadrado}{Barcelona (Spain)}

\sepspace

\NewPart{Education}{}
\noindent

\EducationEntry{PhD, Computer Science}{2013 - 2016}{Université de Lyon}{Thesis: Automatic Role Detection in Online Forums } {IMG/lyon}

\sepspace
\EducationEntry{Master, Artificial Intelligence}{2010 - 2012}{Universitat Politècnica de Catalunya}{Master Thesis: Towards Trust-Aware Recommender Systems} {IMG/UPC}

\sepspace
\EducationEntry{Engineer, Telecommunication}{1999 - 2005}{Universitat Politècnica de Catalunya}{PFC: Sistema experto en la detección y corrección de incidencias en nodos Grid} {IMG/UPC}

\NewPart{Skills \& Abilities}{}


\noindent\SkillsEntry{Machine Learning:}{Recommender Systems, Social Network Analysis, Matrix Factorization,}

\noindent\SkillsEntry{}{Text Mining, Natural Language Processing.}

\noindent\SkillsEntry{Statistics:}{Bayesian Statistics, Monte Carlo methods, Variational Inference, Nonparametric statistics.}

\noindent\SkillsEntry{Tools:}{R, Rcpp, Python, C++, Java, MATLAB, SQL, NoSQL, Linux.}

\noindent\SkillsEntry{Languages:}{ES, CA (native), FR, EN (professional, C2).}


\NewPart{Publications}{}
Some selected publications on several topics:
\begin{itemize}[leftmargin=*]
\item Bayesian Mean-parametrized Nonnegative Binary Matrix Factorization. \\\ \textit{(submitted)}
\item Closed-form Marginal Likelihood in Gamma-Poisson Matrix Factorization. \\\ \textit{International Conference on Machine Learning (2018)}
\item Role detection in online forums based on growth models for trees. \\\textit{Social Network Analysis and Mining (2017)}
\item Non-parametric clustering over user features and latent behavioral functions with dual-view mixture models. \textit{Computational Statistics (2016)}
\item Applying trust metrics based on user interactions to recommendation in social networks. \\\textit{IEEE/ACM International Conference on Advances in Social Networks Analysis and Mining (2012)} 
\item Tecnopolítica: la potencia de las multitudes conectadas. El sistema red 15M, un nuevo paradigma de la política distribuida. \\\textit{Ed.: Universitat Oberta de Catalunya (2012)}
\end{itemize}
\vspace{-0.25cm}             

\NewPart{Academic service}{}
Reviewer for several top tier journals and conferences:
\begin{itemize}[leftmargin=*]
\setlength\itemsep{0.0em}
\item Neural Information Processing Systems (NIPS)
\item IEEE Transactions on Signal Processing (TSP)
\item IEEE International Conference on Acoustics, Speech and Signal Processing (ICASSP)
\item  IEEE International Workshop on Machine Learning for Signal Processing (MLSP)
\end{itemize}
Participation in the organization of scientific workshops and conferences:
\begin{itemize}[leftmargin=*]
\setlength\itemsep{0.0em}
\item \textit{Dynamic Networks workshop.} Institut d'Etudes des Systèmes Complexes de Toulouse (2016)
\item \textit{Advanced mathematics for network analysis}. Institut d'Etudes des Systèmes Complexes de Toulouse (2016)
\item \textit{European Conference on Machine Learning and Principles and Practice of Knowledge Discovery in Databases} (ECML-PKDD, Barcelona 2010)
\end{itemize}
\vspace{-0.25cm}             

\NewPart{References}{}
\begin{tabular}{l l l }
\noindent Cédric Févotte &CNRS, Senior Researcher 
&\small{\faEnvelopeO}~ \normalsize cedric.fevotte@irit.fr \\
\noindent Marie Guégan &Technicolor, Research Scientist 
&\small{\faEnvelopeO}~ \normalsize marie.guegan@technicolor.com\\
\noindent Enrique Garcia-Illera &Telefónica, Product Manager
&\small{\faEnvelopeO}~ \normalsize enrique.garciaillera@telefonica.com
\end{tabular}
\end{document} 

\faGithub {enrique.garciaillera@telefonica.com}
\phone[mobile]{+33~634~266~794}       
%\phone[fixed]{+33~531~986~660}
\email{alberto.lumbreras@irit.fr}                               \homepage{www.albertolumbreras.net}                          \social[linkedin]{albertolumbreras}                        
\social[twitter]{albertlumbreras}                            
\social[github]{alumbreras}
\faEnvelopeO https://github.com/alumbreras